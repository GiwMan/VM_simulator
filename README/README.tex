\documentclass[12pt]{article}
\usepackage[utf8]{inputenc}
\usepackage[greek,english]{babel}
\usepackage{alphabeta}
\usepackage{hyperref}
\hypersetup{
    colorlinks=true,
    linkcolor=blue,
    filecolor=magenta,      
    urlcolor=cyan,
}

\urlstyle{same}



\begin{document}
\title{ergasia2 OS} 
\author{sdi1700076}
\date{\today}
\maketitle

    \begin{itemize}
        \item \textbf{Compilation: make }
        \item \textbf{Execution: ./build/mem\_sim 5 10 1000}
        \item Στον φάκελο \textbf{include} βρίσκονται όλα τα .h αρχεία, στον \textbf{src} τα .c,
        στον \textbf{files} τα αρχεία από τα οποία διαβάζω τα traces και στον \textbf{build} το εκτλέσιμο 
        \textbf{mem\_sim\\}
    \end{itemize}

    Στην \textbf{main2.c} αρχικά γίνεται ο έλεγχος των παραμέτρων που δίνονται από τον χρήστη, γίνονται οι κατάλληλες αρχικοποιήσεις
    και σε περίπτωση σφάλματος τυπώνονται τα αντίστοιχα μηνύματα. Στην συνέχεια, γίνεται ο έλεγος για το αν ο αλγόριθμος που ζητήθηκε είναι ο
    \textbf{LRU} ή ο \textbf{SeconChance (SC).} Ανάλογα με το αποτέλεσμα της παραπάνω σύγκρισης, προχωράω στην υλοποίηση των 2 αλγορίθμων. \\
    Η υλοποίηση των αλγορίθμων αυτών βρίσκεται και αυτή στο main2.c, όπου και στις 2 υλοποιήσεις δημιουργώ 2 hash tables, ένα για κάθε διεργασία,
    διαβάζω αναφορές, με το κατάλληλο shift παίρνω τον ακριβή αριθμός της σελίδας που θέλω σε int, και προχωράω στην εισαγωγή της σελίδας στις δομές. \\


    \textbf{LRU : \\}
    Για τον lru κρατάω σε μια ουρά μια δομή τύπου \textbf{Qnode} η οποία δείχνει κάθε φορά σε έναν επόμενο κόμβο από (page, counter, status), τα οποία
    και κρατάω σε μια δομή \textbf{PageListNode.} Σε μια while, αυξάνω έναν μετρητή \textbf{count} για να γνωρίζω με αυτόν τον τρόπο πότε μπήκε η τελευταία σελίδα.
    Αν η σελίδα υπάρχει στο hash table τότε κάνω \textbf{update} την τιμή του counter στο hash table και ταυτόχρονα πειράζω και τον κόμβο της σελίδας που ήδη υπάρχει
    μεταφέροντάς τον στο τέλος. Αν η σελίδα δεν υπάρχει, αυξάνω τον μετρητή των \textbf{pateFaults} για το hash table και προχωράω στην εισαγωγή της σελίδας στις δομές.
    Σε αυτό το σημείο ελέγχω αν υπάρχει διαθέσιμος χώρος (δεν είναι κατηλειμμένα όλα τα frames). Αν υπάρχει γίνεται η εισαγωγή, αλλιώς
    διαγράφω τον πρώτο κόμβο της ουράς κάθε φορά αφού περιέχει και την μικρότερη τιμή counter. Τέλος, γίνεται η διαγραφή της σελίδας και από τη δομή του πίνακα
    κατακερματισμού. \\ 

    \textbf{SecondChance : \\}
    Για τον SecondChane κρατάω στην δομή SCinfo, τον αριθμό της σελίδας, το bit R/W και το pid. Ομοιώς με τον LRU, με μια for αυτή τη φορά,
    ανάλογα με τον αριθμό της διεργασίας (1/2) εισάγω την πληροφορία που θέλω αρχικά στο hash table και στη συνέχεια στον πίνακά από SCinfo.
    Για αυτή την υλοποίηση δεν έχω να αναφέρω και πολλά καθώς έχω ακολουθήσει παρόμοια λογική με εκέινη που υπάρχει και στο site 
    \url{https://www.programming9.com/programs/c-programs/285-page-replacement-programs-in-c?fbclid=IwAR2Wv_o9Gc9lXA2yjeQrhn-_3EhGwxgmN7anR4Evu7o0rDRl2VmmySV0858}, 
    το οποίο και υπάρχει στα Έγγραφα του μαθήματος. Στο τέλος γίνεται η αποδέσμευση των δομών και υπάρχουν τα μηνύματα που έχουν ζητηθεί.


\end{document}